\documentclass{article}

\usepackage{microtype}

\usepackage{amsmath}
\usepackage{mathtools}

\title{CS225 Spring 2018---Final Project Proposal}
\author{
  Dunncan Enzmann \\ \small{\texttt{git:@MrDJEnz}}
  \and Devin Beckim \\ \small{\texttt{git:@dbeckim}}
  \and Jacob Normyle \\ \small{\texttt{git:@jnormyle}}
  \and Andrew Edwads \\ \small{\texttt{git:@andrewgedwards}}
}
\date{\today}

\begin{document}
\maketitle

\section*{Project: \\Effect Types and Regions --  ATAPL Chapter 3}

We are going to create a project based off stack allocation memory lifetimes. We will design functions which will be allowed to allocate memory on its stack for a value, and then pass a reference to that value to functions called within the body of the function. The point of introducing effect types and regions into OCaml code allows us to use it as an application domain to develop fundamental concepts of effect type systems step by step.

\paragraph{Base Language}

We will be working with simple typed lambda calculus. 
\\
Which include the Terms.
\\Terms (t ::=)
\begin{enumerate}
\item A value expression written as v 
\item A variable written as x. 
\item An application written as t t. 
\item A conditional written as if t then t else t 
\end{enumerate}
Value Expression (v ::=)
\begin{enumerate}
\item An abstraction written as $\lambda$ x.t
\item A truth value written as written as bv.
\end{enumerate}
Truth Values (bv ::=)
\begin{enumerate}
\item True written as tt.
\item False written as ff
\end{enumerate}
With the Evaluation rules as: 
\begin{enumerate}
\item (E-Beta) written as: $(\lambda x.t_{12} ) v_2 \rightarrow [x \rightarrow v_2 ]t_{12}$
\item (E-FixBeta) written as: fix x.t $\rightarrow [ x \rightarrow \verb|fix x.t|]t$
\item (E-IfTrue) written as: if tt then $t_2$ else $t_3 \rightarrow t_2$
\item (E-IfFalse) written as: if ff then $t_2$ else $t_3 \rightarrow t_3$
\item (E-App1) written as: $\frac{t_1\rightarrow\:t'_1}{t_1\:t_2\: \rightarrow t'_1\:t_2}$
\item (E-App2) written as: $\frac{t_2\rightarrow\:t'_2}{v_1\:t_2\: \rightarrow v_1\:t'_2}$
\item (E-If) written as: $\frac{t_1\rightarrow\:t'_1}{if\:t_1\:then\:t_2\:else\:t_{3\:} \rightarrow if\:t'_1\:then\:t_2\:else\:t_{3\:}}$\\
\end{enumerate}

\noindent Also includes the types:
\\Types (T ::=)
\begin{enumerate}
\item Boolean types written as bool.
\item Function type written as T $\rightarrow$ T
\end{enumerate}

With the Typing Rules being:
\begin{enumerate}
\item (T-Var) Written as: $\frac{x\notin \Gamma '}{\Gamma ,x:\:T,\:\Gamma '\:\vdash x\::\:T\:}$
\item (T-Bool) Written as: $\Gamma \:\vdash bv\: : \:bool$ or $\Gamma \:\vdash t_1\: : \:bool$
\item (T-If) Written as: $\frac{\Gamma \:\vdash t_2:\:T\:\:\:\:\:\:\:\:\:\:\Gamma \:\vdash t_{3\:}:\:T}{\Gamma \:\vdash \:if\:t_1\:then\:t_2\:else\:t_3\::\:T}$
\item (T-ABS) Written as: $\frac{\Gamma ,\:x\::\:T_1\:\:\vdash \:t\::\:T_2}{\Gamma \:\:\vdash \:\lambda x.t\::\:T_1\:\:\rightarrow \:T_2}$
\item (T-App) Written as: $\frac{\Gamma \:\vdash t_0\::\:T_1\:\:\rightarrow \:T_2\:\:\:\:\:\:\:\:\Gamma \:\vdash \:t_1\::\:T_1}{\Gamma \:\vdash \:t_0\:\:t_1\::\:T_2}$
\item (T-Fix) Written as: $\frac{\Gamma ,\:x\::\:T\:\vdash \:t\::\:T}{\Gamma \:\vdash \:fix\:\:x.t\::\:T}$
\end{enumerate}

\paragraph{Extended Language}

Extend this language with tagging and un-tagging terms.
\\This consists of one new type (T::=) written as:
\begin{enumerate}
\item A tagged value type, written T at p
\end{enumerate}
Two new terms (t ::=)
\begin{enumerate}
\item Type tagging, written t at p
\item Type untagging, written t ! p
\end{enumerate}
One new value expressions (v ::=):
\begin{enumerate}
\item Tagged Value: $<v>_{\rho }$
\end{enumerate}
One new label expressions (p ::=):
\begin{enumerate}
\item Tagged Value: $\rho$
\end{enumerate}
New Evaluation Rules:
\begin{enumerate}
\item (E-Tag) Written as: $\frac{t\:\rightarrow \:t'}{t\:at\:\rho \:\rightarrow t'\:at\:\rho \:}$  
\item (E-TagBeta) Written as: $v\:at\:\rho \:\rightarrow \:<v>_{\rho }$
\item (E-UnTag) Written as: $\frac{t\:\rightarrow \:t'}{t\:!\:\rho \:\rightarrow t'\:!\:\rho \:}$
\item (E-UnTagBeta) Written as: $<v>_\rho$ ! $\rho \rightarrow v$
\end{enumerate}

%these are example notes from the given template...
\paragraph{Applications}
$\newline \newline$
Label
\begin{enumerate}
\item Labels let us distinguish values in different places although they are extensionally equal.
\item For purposes, labels can be though of as "regions".
\end{enumerate}

\noindent Tag/UnTag Variables
\begin{enumerate}
\item The tagging and untagging operations serve to name certain sets of values and to mark where such values are constructed and used.
\item Note:
	multiple sub terms of a term may have the same label.
	Even though a label may occur once in a program it may tag multiple values at run time.
\end{enumerate}
Typing based on terms:
\begin{enumerate}
\item Consistent with our material, every term has a "type" (and a label) and operations are restricted to terms of a certain type.
\end{enumerate}

\noindent Constructor/De-constructor completion
\begin{enumerate}
\item A completion where each expression is tagged and untagging takes place.
\end{enumerate}


\paragraph{Project Goals}

For this project, we plan to complete:
\begin{enumerate}
\item Type existing variables via explicit tagging and untagging operations
\item Create outline for reference-passing function designed to allocated memory on a stack by using new effect types for system variables.
\item We hope to get to implement the effect type judgments by using an effect expression, if there is an empty effect we write it as $\emptyset$, and $^\emptyset$T is abbreviated to T. 
\item The typing rule may depend not only on the results of evaluations, but on certain aspects of the evaluation itself, in other words on how a value is computed, not just
which value is computed. To capture properties of evaluation we will introduce
effects.
\end{enumerate}

\paragraph{Expected Challenges}

We expect that implementing effects into our call by value language will cause some issue in that capturing properties of the evaluation, since there cannot be any judgments using a deleted or accessible region or free variables, rather we need to denote $\Gamma \vdash t : ^\phi T$ to express free variables are going to be bound to the values of the types. Rather than the type of it self.

\paragraph{Timeline and Milestones}

By the checkpoint we hope to have completed:
\begin{enumerate}
\item A complete writeup of this document.
\item A plan for a comprehensive list of test cases for our new terms and types.
\item Research more into Effect Typed Languages.
\end{enumerate}

\noindent
By the final project draft we hope to have completed:
\begin{enumerate}
\item Full implementation of the test cases of the newly introduced terms types
\item Having all the tests passing the suite.
\item Having a fully comprehensive working program.
\item A draft writeup that explains the on-paper formalism of our
  implementation
\item A draft of a presentation with 5 slides as the starting point for our
  in-class presentation
\end{enumerate}

\noindent
By the final project submissions we hope to have completed:
\begin{enumerate}
\item The final writeup and presentation
\item Any remaining implementation work that was missing in the final project
  draft
\end{enumerate}



\end{document}

